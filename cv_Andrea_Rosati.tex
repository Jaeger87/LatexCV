% !TeX program = xelatex

%%%%%%%%%%%%%%%%%%%%%%%%%%%%%%%%%%%%%%%%%
% Friggeri Resume/CV
% XeLaTeX Template
% Version 1.2 (3/5/15)
%
% This template has been downloaded from:
% http://www.LaTeXTemplates.com
%
% Original author:
% Adrien Friggeri (adrien@friggeri.net)
% https://github.com/afriggeri/CV
%
% License:
% CC BY-NC-SA 3.0 (http://creativecommons.org/licenses/by-nc-sa/3.0/)
%
% Important notes:
% This template needs to be compiled with XeLaTeX and the bibliography, if used,
% needs to be compiled with biber rather than bibtex.
%
%%%%%%%%%%%%%%%%%%%%%%%%%%%%%%%%%%%%%%%%%

\documentclass[]{friggeri-cv} % Add 'print' as an option into the square bracket to remove colors from this template for printing

\addbibresource{bibliography.bib} % Specify the bibliography file to include publications

\begin{document}

\header{Andrea}{Rosati}{Software developer} % Your name and current job title/field

%----------------------------------------------------------------------------------------
%	SIDEBAR SECTION
%----------------------------------------------------------------------------------------

\begin{aside} % In the aside, each new line forces a line break
\section{Contact}
+39 3332232755 {\color{gray} {\faPhone}}
~
\href{mailto:rosati.1595834@gmail.com}{rosati.1595834@gmail.com} {\color{gray} {\faEnvelope}}
\href{http://jaegerbox.net}{www.jaegerbox.net} {\color{gray} {\faHome}}
\href{https://github.com/Jaeger87}{Jaeger87} {\color{gray} {\faGithub}}
\href{https://it.linkedin.com/in/andrea-rosati-050091a7}{Andrea Rosati} {\color{gray} {\faLinkedin}}
\section{Languages}
italian mother tongue
English \& Spanish fluency
\section{Programming}
{\color{red} $\varheartsuit$} Java
C\#, Python, Processing
Scala, Arduino, SQL
Android, \LaTeX
\section{Communities}
V++, Hebocon organizers,
\href{http://muglab.uniroma3.it/}{MUG Roma Tre}, Codemotion tech communities 
\end{aside}



%----------------------------------------------------------------------------------------
%	EDUCATION SECTION
%----------------------------------------------------------------------------------------

\section{Education}

\begin{entrylist}

%------------------------------------------------

\entry
{2013--2017}
{Bachelor {\normalfont of Computer science 105\backslash110}}
{Sapienza University of Rome, Italy}
{Thesis title: \emph{Hint keeper: An unsolicited hint system for serious games} \\I designed and developed a hint system for serious games. It analyzes player data and emotions and can decide to provide hints to players who are in trouble in the game.}

%------------------------------------------------

\end{entrylist}

%----------------------------------------------------------------------------------------
%	WORK EXPERIENCE SECTION
%----------------------------------------------------------------------------------------

\section{Experience}

\begin{entrylist}

%------------------------------------------------

\entry
{Feb 2018 --\\ May 2019}
{\href{http://rfidlab.di.uniroma1.it/}{Sapienza university - RFID Lab}}
{Rome, Italy}
{Research fellow. Android, Java, C\# Aspnet developer\\
	I'm working on an IOT project about a smartwatch, a web application based on the Spring framework and the maintenance of two Android apps.}

\entry
{Sep 2018 --\\ Nov 2018}
{\href{https://www.makinarium.it/}{Makinarium}}
{Rome, Italy}
{Animatronic - Android\\
	I worked on a animatronic project for the Italian movie "{\href{https://www.imdb.com/title/tt9269662/}{Mollami}}", for this project i developed a control system based on two physical controllers and one Android app.}


\entry
{2019}
{\href{https://tree.it/}{Tree srl}}
{Rome, Italy}
{Java Teacher\\
	i gave the first lesson about Java and OOP for the \href{https://ytia.it/}{young talent in action} Java course.
} 


\entry
{2017 -- Now}
{\href{https://codemotionkids.com/}{Codemotion kids}}
{Rome, Italy}
{Kids robotics teacher\\
	My activity with codemotion kids consists in holding programming basics laboratories for kids (from 6 to 14 years old) during some important events such as the \href{http://www.rainews.it/dl/rainews/media/Maker-Faire-giovani-inventori-presentano-progetti-innovativi-i-bimbi-vanno-su-Marte-d40b391c-1db8-4a57-85d2-63dc5feb860b.html}{2017 Maker faire Rome}. During these labs we use  \href{https://www.makeblock.com/steam-kits/mbot}{mBot robots} with \href{https://scratch.mit.edu/}{Scratch}. 
}

\entry
{Sep 2017 --\\ Nov 2017}
{\href{http://www.staersistemi.com/it}{Staer sistemi}}
{Rome, Italy}
{Aspnet-C\# Angular fullstack developer\\ I was involved in a district security project.	
}

\entry
{Feb 2015 --\\ Feb 2016}
{\href{https://www.ptvgroup.com/it/}{PTV SISTeMA}}
{Rome, Italy}
{Aspnet-C\# backend developer\\ 
	The project I worked on concerned a web application for managing the mobility of employees in large companies.
}

\end{entrylist}

%----------------------------------------------------------------------------------------
%	AWARDS SECTION
%----------------------------------------------------------------------------------------

\section{Projects}

\begin{entrylist}

%------------------------------------------------

\entry
{2014}
{\href{https://www.facebook.com/Open-Basiligotchi-792468080818688/?rc=p}{Open Basiligotchi}}
{Rome, Italy}
{Arduino - Android\\
	Open Basiligotchi is an Arduino based project able to transform a plant in a tamagotchi. It was born during a \href{http://roma.startupitalia.eu/20041-20140331-basilico}{hackathon} and has been presented in the 2014 Rome Maker Faire obtaining a surpraising success and the attention of the \href{https://youtu.be/M7RPfu6zRww}{media}. Lately the Arduino/Android code with the electronic schema were released under a  \href{https://github.com/Jaeger87/OpenBasiligotchi}{open source license}.
}

%------------------------------------------------
\entry
{2014}
{Jargon}
{Rome, Italy}
{Java\\
	Jargon is a 2D multiplayer videogame for Android and PC developed in java using the \href{https://libgdx.badlogicgames.com/}{libGDX} framework.
	It was born as a university project for a course and despite was completed we never released it on Google playstore.
	In particular I worked on the netcode and the game's physics.	
}


\entry
{2017}
{\href{https://globalgamejam.org/2017/games/playmobil-ma}{Playmobil M***A}}
{Rome, Italy}
{Processing\\
	This game was created during the 2017 Global Game Jam. It has a particular control system in which a player moves a frog by physically jumping. The game gained a good success and the italian version of 
	\href{https://motherboard.vice.com/it/article/aejzz8/global-game-jam-2017-giochi-italiani}{Motherboard Vice} talked about it.
}



\entry
{2015 -- Now}
{\href{http://jaegerbox.net/botticelli-index/}{Botticelli}}
{Rome, Italy}
{Java\\
	Botticelli is an opensource java framework for rapid development of \href{https://telegram.org/}{Telegram} chat bots. The project is managed for entirely by me.	
}


\entry
{2016}
{\href{http://jaegerbox.net/third-prize-intel-iot-roadshow-andromeda/}{Andromeda}}
{Rome, Italy}
{Java\\
	Andromeda was born during the Intel IOT roadshow, an hackathon hosted by Intel for promoting the Galileo 2 board. It is a \href{https://youtu.be/BzG7xg5_9JM}{smart rack designed for the bike sharing} connected to the internet. The project was really appreciated and my team won the third prize of the competition. 
}

\entry
{2015 -- Now}
{\href{https://www.facebook.com/HeboconRoma/}{Hebocon Roma}}
{Rome, Italy}
{Event\\
	Hebocon is a robot wars competition for those technically ungifted. This contest was born in Japan and it was a worldwide success that even \href{https://makezine.com/2017/05/08/hebocon-crappy-robot-competition/}{Make magazine} talked about it. Together with a friend we brought the hebocon to Rome and we created new competitions like the \href{https://www.youtube.com/watch?v=SF5oQEZsgzA}{paint edition} that  was even hosted in \href{http://portal.nifty.com/kiji/161011197795_1.htm}{Japan}.
}


\end{entrylist}

%---------------------------------------------------------------------------
%----------------------------------------------------------------------------------------


%	INTERESTS SECTION
%----------------------------------------------------------------------------------------

\section{Interests}

\textbf{professional:} Videogames development, embedded systems, IOT, UX, networking, hackathon, data analysis, software design, education. \textbf{personal:} Videogames, travels, movies, books, boardgames, DIY, tv shows, music, digital art, Reddit.


\end{document}